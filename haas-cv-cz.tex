%%%%%%%%%%%%%%%%%%%%%%%%%%%%%%%%%%%%%%%%%
% Friggeri Resume/CV
% XeLaTeX Template
% Version 1.2 (3/5/15)
%
% This template has been downloaded from:
% http://www.LaTeXTemplates.com
%
% Original author:
% Adrien Friggeri (adrien@friggeri.net)
% https://github.com/afriggeri/CV
%
% License:
% CC BY-NC-SA 3.0 (http://creativecommons.org/licenses/by-nc-sa/3.0/)
%
% Important notes:
% This template needs to be compiled with XeLaTeX and the bibliography, if used,
% needs to be compiled with biber rather than bibtex.
%
%%%%%%%%%%%%%%%%%%%%%%%%%%%%%%%%%%%%%%%%%

\documentclass[]{friggeri-cv} % Add 'print' as an option into the square bracket to remove colors from this template for printing

\begin{document}

\header{Antonín }{Haas}{opensource nadšenec \& vývojář} % Your name and current job title/field

%----------------------------------------------------------------------------------------
%	SIDEBAR SECTION
%----------------------------------------------------------------------------------------

\begin{aside} % In the aside, each new line forces a line break
\section{Kontakt}
Okružní~255
Jevíčko,~569~43
Czech Republic
~
+420~776~053~847
\href{mailto:tondahaas@gmail.com}{tondahaas@gmail.com}
~
\href{https://github.com/tuxanek}{github.com/tuxanek}
~
\href{https://flickr.com/tondahaas}{flickr.com/tondahaas}
\section{Jazyky}
Český jazyk (rodilý mluvčí)
Anglický jazyk (B2)
\section{Programování}
%{\color{red} $\varheart$} Python $\varheartsuit$
%{\color{red} $\varheartsuit$} Java, 
Java, C/C++, C\#, Python
HTML, CSS
LaTeX, Bash, Git
\section{Software}
Debian, Arch
\section{Řidičský průkaz}
ne
\end{aside}

%----------------------------------------------------------------------------------------
%	EDUCATION SECTION
%----------------------------------------------------------------------------------------

\section{Vzdělání}

\begin{entrylist}

\entry
{2015--dosud}
{Magisterské}
{Univerzita Palackého, Olomouc}
{\emph{Steganografické metody} \\ Cílem práce je určit využití fuzzy logiky ve steganografii.}

\entry
{2012--2015}
{Bakalářské}
{Univerzita Palackého, Olomouc}
{\emph{Syntaktická analýza shora dolů} \\ Práce demonstuje deterministickou syntaktickou analýzu metodou shora dolů pro bezkontextové gramatiky.}

\entry
{2008--2012}
{{\normalfont Gymnázium Jevíčko}}
{Gymnázium Jevíčko}

\end{entrylist}

%----------------------------------------------------------------------------------------
%	WORK EXPERIENCE SECTION
%----------------------------------------------------------------------------------------

\section{Pracovní zkušenosti}
\subsection{Brigády}
\begin{entrylist}

\entry
{2013--dosud}
{Univerzita Palackého}
{Olomouc, Czech Republic}
{\emph{Správce sítě} \\
Technická podpora pro studenty na kolejích. Konfigurace síťových rozhraní na různých OS, instalování ovladačů.}

\entry
{2015--2016}
{CUTTER Systems spol. s r.o.}
{Prostějov, Czech Republic}
{\emph{Firmware Developer} \\
Vývoj FW pro různé mikroprocesory (PIC, AVR, ARM) v jazyce C. Síťová komunikace, zpracování sensorových dat, routování, optimalizace pro zařízení na baterie atd.}

\end{entrylist}

%----------------------------------------------------------------------------------------
%	PROJECTS SECTION
%----------------------------------------------------------------------------------------
\section{Projekty}

\begin{entrylist}

\entry
{2016--dosud}
{\href{https://tux.inf.upol.cz}{Tux}}
{Univerzita Palackého, Olomouc}
{Správa linuxového studentského serveru}

\entry
{2012--2015}
{Ateneo}
{Univerzita Palackého, Olomouc}
{Sborový zpěv: bas.}

\entry
{2013, 2015}
{Redbot}
{Brno, Czech Republic}
{Účast v programovací soutěži Red Hat.}

\entry
{2012}
{AMAVET}
{Prague, Czech Republic}
{Středoškolský přírodovědecký projekt: Konstrukce lineární pumpy (titrátor). Účast v celorepublikovém kole.}

\end{entrylist}

%----------------------------------------------------------------------------------------
%	INTERESTS SECTION
%----------------------------------------------------------------------------------------

\section{Zájmy}

\textbf{Profesní:} progrramování, GNU/Linux, opensource, Raspberry Pi
\\
\textbf{Osobní:} hudba, sborový zpěv, analogová \& digitální fotografie

%----------------------------------------------------------------------------------------

\end{document}
